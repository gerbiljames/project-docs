\chapter{Introduction}
\label{cha:Intro}

Most of us use the internet every day, but do we really understand what it is? It was once described as being "a series of tubes", which, while incorrect, is not far from the truth. The internet itself is effectively a construct which has arisen through the \textit{inter}connection of smaller computer \textit{net}works and exists only as a whole because these networks required connections to others to communicate information. More and more networks required these connections and eventually it became possible to communicate with most other networks by forwarding information across many others. 

However, networks were often set up in places which did not offer another network for a connection and a more rigid way of connecting to the internet was required. Companies specialising in providing connection to the internet began appearing, these were the first Internet Service Providers (ISPs). An ISP offers an internet connection as a service, providing an access point through which to route data, or traffic, to the rest of the internet. There were of course very few ISPs when the internet first began, as the initial investment in laying the physical cables and equipment necessary was very large, many of these first ventures were government backed and publicly funded. 

When a business or person wishes to connect their network to the internet they must purchase the ability to do so from an ISP that operates in their area, sometimes they will have a choice, sometimes not. That ISP then must forward any traffic which that customer generates to the correct network on the internet; this is a \textit{provider-customer} relationship. But what if that ISP does not have a direct connection to this network? It is insufficient to simply fail to deliver the traffic, customers would not accept that! Instead it is the prerogative of the ISP to provide as complete traffic forwarding capabilities as possible. ISPs accomplish this through peer agreements with other ISPs, where each ISP agrees to forward the others traffic if it can. This is a \textit{peer-peer} relationship and is the reason for the widespread adoption of the internet as a communication medium. It is important to consider that ISPs must have as many peer agreements as they can, given their physical location, as this makes them more attractive prospects for new networks. 

\section{Structure of the Internet}

Considering how the internet is structured may seem like a difficult question at first, but the structure is actually very simple. If a small business wishes to connect themselves to the internet then they make an agreement with a local ISP in order to do so. This agreement is effectively the same one the ISP makes with all their customers, which may number from a few to millions. If we take an ISP with a million customers, for example, then they must connect to another ISP which allows them to forward all traffic generated by these customers to the correct network. However, as these peers also connect to peers, this reduces the number of peer-peer connections required to very low numbers, way below the number of provider-customer links. 

In truth, all ISPs do not always peer with others, instead there is a hierarchy of sorts through which \textit{lower-level} ISPs pay \textit{higher-level} ones to forward their traffic in the same way a business would a their ISP. These are still provider-customer relationships but they serve to link ISPs with others that are far away physically, through an intermediate ISP. This type of internetworking causes the internet to have a very predictable and interesting structure. 

ISPs which reside at the top of the hierarchy have connections to other large networks in order to provide long range transit for large amounts of traffic. Those in the middle tiers often sell links between higher tier ISPs and lower tier ones, but they will always have more customers than suppliers, meaning they will link many low-tier ISPs to a single high-tier one. Low-tier ISPs will often link many small networks to a single mid-tier network. At the very bottom tier a network will be connected to a single ISP. It is important to note that the internet does not have a rigid tier system, this is simply an emergent property of the way networks are interconnected.

If the internet were to be represented as a tree, then it would have an extremely high branch factor, which causes it to be very difficult to effectively visualise. Though, strictly speaking, the internet cannot be a tree because it has no defined root node and there is no concept of parent and child in a peer-peer link. However, the internet does have a very densely linked core, where few, high degree nodes are connected to others of the same kind. In addition, moving out of the core and down the \textit{tree} reveals many smaller nodes which only have a single, or very few, connections. Clearly, the further from the core a node is the more nodes exist \textit{near} it, in practice this is actually an exponential increase and it is very difficult to visualise graphs such as this due to there not being enough space for the low-tier systems to fit. This begs the question; is there a way of visualising the internet which preserves this hierarchical structure?

\section{Aims}

The aim of this project is to construct a graph of the internet and embed it into a space which supports its properties, namely a hyperbolic space. The project will effectively consist of the following \textit{sub-aims}:

\begin{itemize}
	\item{Produce an embedding of the internet, or a subset thereof, in hyperbolic space at the level of Autonomous Systems.}
	\item{Provide a method to dynamically navigate this embedding.}
\end{itemize}

\section{Statement of Ethics}
