\chapter{Further Work}

This project has shown that it is indeed possible to produce a hyperbolic embedding of the internet, at the level of ASes which visualises information that cannot be shown in Euclidean space. Further, this project made use of geographical location data for ASes to produce this embedding; something which has never been done before. There is, however, scope for further work in this problem domain which this chapter aims to outline. 

An obvious improvement to the embeddings would be the ability to embed the entire internet into hyperbolic space and produce visualisations thereof. This was not possible in this project due to the large memory requirement of such a task. Indeed, the entire internet consists of over 50,000 ASes according to data collected in this project and generating the visualisations for a dataset of just over 1000 requires approximately 2GB of RAM. Perhaps it would be possible to modify the embedding method so that a distance matrix is not required and coordinates can be used instead.
However, the existing method should have no trouble scaling up to this level, it just requires far more storage and processing power to do so.

The modification made to the distance calculation in section \ref{sec:adjusting} used data which had already been collected to augment the current calculation. Perhaps it would be possible to use other data to augment this calculation differently. The degree separation of nodes does not necessarily represent the tiering difference of those nodes and it may be possible to determine some sort of concrete tiering from other data. If such a thing were possible then it would be an ideal replacement for the method used in this project. Considering this further, any data which has a defined \textit{distance} could be used to further improve the visualisations.

In section \ref{sec:adding_links}, in which links were added to the visualisations, a problem was encountered which meant ASes locations were not what they should be. As stated previously, this is due to the method used to locate ASes being tied to IP address, which may not accurately represent their real location. Instead of using this method, it may be possible to engineer a new method for determining the location of an AS using other data. If such a method could be implemented, it could be used to improve the quality of the embeddings in this project drastically. 