\chapter{Literature Review}
\label{cha:LitReview}

\section{Preliminaries}
\label{sec:LitReviewPrelim}

\section{Hyperbolic Space}
\label{sec:LitReviewHyperbolicSpace}

Hyperbolic space (or, more broadly, hyperbolic geometry) is by no means a recent concept and was postulated by Carl Friedrich Gauss in 1824 \cite{ratcliffe_foundations_2006}. In order to understand what hyperbolic geometry is, it is first necessary to review what is often described as standard: Euclidean geometry.

Euclidean geometry (postulated by Euclid in around 300 B.C. \cite{ratcliffe_foundations_2006}) is based upon five postulates, which Euclid described as follows \cite{ratcliffe_foundations_2006}:
\begin{enumerate}
\item \textit{A straight line may be drawn from any point to any other point.}
\item \textit{A finite straight line may be extended continuously in a straight line.}
\item \textit{A circle may be drawn with any centre and any radius.}
\item \textit{All right angles are equal.}
\item \textit{If a straight line falling on two straight lines makes the interior angles on the same side less than two right angles, the two straight lines, if extended infinitely, meet on the side on which the angles are less than two right angles.}
\end{enumerate}

The first four of these postulates seem both simple and natural, however the fifth is more involved and it is not immediately apparent what is being described by it. Euclid's fifth postulate effectively states that given a straight line and a point, there is only \textbf{one} straight line which passes through the point and is parallel to the original straight line. 

When considering the first four postulates it is easy to see that they should simply be accepted without proof due to their simplicity. However, the fifth postulate cannot be so easily accepted. A solution to this is to attempt to prove the fifth postulate using the others, something which was attempted for over two thousand years \cite{ratcliffe_foundations_2006} but was not possible. The only remaining hypothesis is that the fifth postulate is not required for a definition of geometry!

This seems absurd, as it is relatively simple to demonstrate in two dimensional Euclidean geometry that the fifth postulate holds, using only a piece of paper and a pencil. However, this does not prove that the postulate holds in all cases, only in this one instance. In order to refute the postulate, it is necessary cast out this rigid view of a a flat plane and to consider other geometries in which to work. 

If, instead of a plane, all geometry took place on the surface of a sphere then very different behaviour is observed. Of course, the surface of the sphere is a two dimensional plane but with one key difference: it is finite. On this finite plane a straight line is any which is a great circle of the sphere. In this case, it is clear that any straight line must have \textbf{zero} straight lines which are parallel to it, as all great circles must meet at exactly two points. 