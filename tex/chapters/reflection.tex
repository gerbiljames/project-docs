\chapter{Reflection}
When I first began this project I had never heard of hyperbolic space and had no idea what an embedding was. Learning about this entirely new way of thinking about geometry has been both enlightening and enjoyable. At the beginning, I was unsure I would be able to complete this project as the topic seemed so complex, but I did my best to bring together the knowledge required to do so. Upon finishing, I feel that I have produced something which genuinely contributes to this field and contains original thought.

During development, things did not always go to plan. For example, finding a source for the AS location data proved difficult. It was not possible to directly locate each AS, instead an approximate location had to be computed from the IP addresses associated with it. I would have been happier with a direct location method, but I made the best of the data that was available. The embedding and visualisation solution itself does not discriminate over data sources, however, and would adapt readily to more accurate location detection.

As my background is in the more practical elements of Computer Science, I often found the mathematical elements (of which there were many!) of this project challenging. The learning curve was steep at first, but I was able to gain an understanding of many new concepts in such a short time. Given a chance to go back and change I would pick this project again in every case.

I feel that the culmination of this project was the visualisation produced in figure \ref{fig:distance_new_klein} which was able to show the relative importance of an AS while preserving geographical data. This was one of my first aims for the project upon reading the specification. I hoped to produce something which had a practical use and I feel that I have succeeded.

This project has been difficult, and I would not have it any other way. I just hope it helps to further our understanding of what the internet is and how it is changing.